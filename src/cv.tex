\documentclass[hidelinks__VERSION__]{adamyi-cv} 

\begin{document}

\revision{__REVISION__}
\footertext{This version was compiled on \today\ at \DTMcurrenttime. Source code available at \href{https://github.com/adamyi/cv}{https://github.com/adamyi/cv} under MIT license.}

\header{Adam}{Yi}{full-stack developer, white hat hacker and enthusiastic researcher}

%----------------------------------------------------------------------------------------
%	SIDEBAR SECTION
%----------------------------------------------------------------------------------------

\begin{aside} % In the aside, each new line forces a line break
\section{contact}
+61 4 3207 0150
+1 (402) 417-0115
+1 (808) 278-0103
+86 186 1071 3116
~
\href{mailto:i@adamyi.com}{i@adamyi.com}
\href{https://yiad.am}{https://yiad.am}
~
\href{https://github.com/adamyi}{GitHub: adamyi}
\href{https://www.linkedin.com/in/adam-yi/}{LinkedIn: adam-yi}
\href{https://www.facebook.com/adamxuanyi}{Facebook: adamxuanyi}
\href{https://stackoverflow.com/users/6482303/adamyi}{Stack Overflow: adamyi}
\section{programming}
~
C, C++, PHP
Java, Golang, Python
Swift, Objective-C
Matlab, \LaTeX, Shell
Kotlin, JavaScript
CSS3 \& HTML5
\versionsection
\end{aside}

%----------------------------------------------------------------------------------------
%	EDUCATION SECTION
%----------------------------------------------------------------------------------------

\section{education}

\begin{entrylist}

%------------------------------------------------

\entry
{\href{https://unsw.edu.au}{University of New South Wales}, Sydney Australia}
{Jul. 2018 -- Sep. 2021}
{\emph{\textbf{B.S.} in Computer Science}\\
CS course WAM: 97.5/100, relavent courses: \emph{Introduction to Programming} (96/100 HD), \emph{Data Structures and Algorithms} (99/100 HD), \emph{System and Software Security Assessment} (unofficial student), \emph{Digital Forensics and Incident Response} (unofficial student), \emph{Software Engineering Fundamentals} (currently taking), \emph{Web Application Security and Testing} (currently taking), \emph{Database Systems} (currently taking)
}

\entry
{\href{http://beijing101.com}{Beijing 101 High School}, Beijing China}
{Sep. 2012 - Jul. 2018}
{\emph{\textbf{Qian Xuesen} Experimental Class}\\
Relevant positions: Founding Director of the Technology Department of the Student Union
}

\end{entrylist}

%----------------------------------------------------------------------------------------
%	RESEARCH SECTION
%----------------------------------------------------------------------------------------

\section{research}

\begin{entrylist}

%------------------------------------------------

\entry
{\href{http://cmuchimps.org/}{CHIMPS Lab}, \href{https://www.hcii.cmu.edu/}{Human-Computer Interaction Institute}, \href{https://www.cmu.edu/}{Carnegie Mellon University}}
{Dec. 2016 -- Present}
{\emph{Research Assistant}
\begin{itemize}
\item Came up with the core architecture and co-developed \href{https://github.com/MessageOnTap}{MessageOnTap}, an Android framework for developers to develop IM plugins, which will be open sourced in a few months, as one of the 4 core members in a team of 12 instructed by PhD student \href{mailto:fanglin@cmu.edu}{Fanglin Chen}.
\item Developed an online system and hired 2,500+ Amazon mTurk workers to annotate sensitive information in 65,000+ pieces of data for us to train a deep learning PII (Personally Identifiable Information) scrubbing algorithm (DeepScrub).
\item Received a GPU donation worth \$1,200 from \href{https://www.nvidia.com/}{NVIDIA Corporation} to support myself to conduct research in NLP (Natural Language Processing).
\item Trained a Word2Vec model with Reddit data to predict apps user might use.
\item Co-designed a semantic-structured graph database (Pocket) for mobile personal data.
\item Run the internal Git, Maven \& Jenkins server used by our team.
\end{itemize}}

%------------------------------------------------

\end{entrylist}

%----------------------------------------------------------------------------------------
%	AWARDS SECTION
%----------------------------------------------------------------------------------------

\section{awards}

\begin{entrylist}

%------------------------------------------------

\entry
{Second Place}
{2018}
{\emph{\textbf{Cyber Security Challenge Australia (CySCA)}}\\
CySCA is Australia’s only national cybersecurity competition (CTF), run by The Federal Government’s Australian Cyber Security Centre and AustCyber, Telstra, PwC, Cisco, Microsoft, Commonwealth Bank, Splunk, BAE Systems, and HackLabs. I participated in UNSW3, a team of 4 students, and got the second place among 109 teams across Australia (427 students). The award includes flights and accommodation to Kiwicon 2018 in Wellington, New Zealand, and an electronic device of my choice.
}

%------------------------------------------------

\entry
{First Place}
{2018}
{\emph{\textbf{UNSW Security Society Capture the Flag}}\\
It's UNSW Security Society's largest internal CTF competition. As a first-time CTF player, I got first place out of 30 teams.
}

%------------------------------------------------

 \entry
{Second Prize}
{2016}
{\emph{\textbf{Awarding Program for Future Scientists (China)}}\\
It is the most recognized, official and advanced science innovation competition for teenagers in China, organized by China Association for Science and Technology, Chinese Academy of Science, Chinese Academy of Engineering, National Natural Science Foundation of China, and H.S. Chau Foundation. 1500 students applied. 100 are chosen for the final round by 160 professors. They are then graded by a committee of 60 academicians, and selected for 15 first prizes, 35 second prizes \& 48 third prizes. I won the second prize for designing a catkin collecting device to mitigate the severe air pollution in China.
}

%------------------------------------------------

\entry
{Gold Medal}
{2015}
{\emph{\textbf{International Exhibition of Inventions of Geneva}}\\
The exhibition was organized by the Swiss Federal Government, during which I received a gold medal for designing a particulate matter detector using the light scattering method during junior/middle school. During the exhibition, I also received a Special Award from the \href{http://www.tia-tw.net/}{Taiwan Invention Association}.
}

%------------------------------------------------

\end{entrylist}

%----------------------------------------------------------------------------------------
%	PROJECTS SECTION
%----------------------------------------------------------------------------------------

\section{projects}

\begin{entrylist}

%------------------------------------------------

\entry
{White Hat Hacking (Ethical Hacking/Cybersecurity Research)}
{Feb. 2015 -- Present}
{\begin{itemize}
\item Discovered 10+ vulnerabilities in various bug bounty programs like Atlassian's.
\item Received 4,000 CNY and a Huawei Smart Bracelet as award for discovering a zero-day vulnerability of eYou, one of the two most widely-used email server software in China.
\end{itemize}}

%------------------------------------------------

\entry
{Google Developers Community Programs}
{Jun. 2016 -- Present}
{\emph{Translator (Volunteer)}
\begin{itemize}
\item Translated English subtitles of Google Developer videos including I/O keynotes into Chinese for official Youtube, Youku and Bilibili.
\item Invited to Google I/O in San Jose as a Community Partner.
\item Reviewed and translated some of Google's official blog posts and announcements before releasing.
\item Volunteered at/Co-organized several local events like TensorFlow Dev Summit 2018 Extended Beijing at Google Beijing Office.
\item Co-developed Flutter's official website in China (flutter-io.cn).
\end{itemize}}

%------------------------------------------------

\entry
{Bojangles}
{Aug. 2018 -- Present}
{\emph{Developer}
\begin{itemize}
\item Co-developing a tool to help UNSW students plan their course schedule in a team of 4 within the \href{https://www.csesoc.unsw.edu.au/}{CSE Society}.
\item In charge of the algorithm to generate possible timetables, and came up with a PoC with a Genetic Algorithm.
\end{itemize}}


%------------------------------------------------

\entry
{\href{https://www.munpanel.com}{MUNPANEL}}
{Nov. 2016 -- Jul. 2017}
{\emph{Founder \& Chief Technology Officer}
\begin{itemize}
\item Came up with the idea to design an online Model UN system that connects all conferences, clubs, and MUNers together by facilitating oraganizers and participants of different conferences and clubs with one unified system.
\item Led a team of 4 members to develop and run the Model UN platform, which has served 3 conferences, 170+ registered schools, and 2,000+ users, with 500+ maximum DAU (daily active users), 235,000+ total visits and 490,000+ CNY cash flow till July 2017.
\end{itemize}}

%------------------------------------------------

\entry
{\href{https://bjmun.cn}{Beijing Model United Nations Association for High School Students}}
{Aug. 2016 -- Aug. 2017}
{\emph{Director of Academics (English)}
\begin{itemize}
\item Managed the largest MUN association in Beijing (60+ member schools) with 12 other co-organizers.
\item Took charge of all facets of English committees in two largest Model UN conferences for high school students in Beijing, such as reviewing academic design, background guide, etc.
\item Run the web, email, ns \& git server of the association.
\end{itemize}}

%------------------------------------------------

\entry
{\href{https://chaoli.club}{Chaoli Forum}}
{Nov. 2014 -- May. 2015}
{\emph{Co-Founder \& Developer}
\begin{itemize}
\item Developed functions like documents uploading, CDN integration, customer feedback, etc. based on esoTalk, a no-longer maintained open source forum program, as a member of the initial development team and founding team of Chaoli forum, a forum for Math, Physics, Chemistry, and Informatics lovers, which now has 5,000+ users and 40,000+ threads.
\end{itemize}}

%------------------------------------------------



\end{entrylist}

\end{document}
